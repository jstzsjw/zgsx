%%%%%%%%%%%%%%%%%%%%%%%%%%%%%%%设置版面控制%%%%%%%%%%%%%%%%%%%%%%%%%%%%%%%%%%
\documentclass[a4paper, 12pt, notitlepage, onecolumn, oneside ,fleqn ,twoside]{book}
\usepackage[colorlinks,
            linkcolor=red,
            anchorcolor=blue,
            citecolor=green,
            bookmarks]{hyperref}
            \hypersetup{pdftitle={曾哥考研数学班课堂例题},pdfauthor={帅靖文},pdfkeywords={考研、数学、曾哥},pdfsubject={考研数学},pdfstartview={FitV}}
%\usepackage[top=1.2in,bottom=1.2in,left=1.2in,right=1in]{geometry}                          % 设置版面,纸张空距
\usepackage{indentfirst}                 % 用于首行缩进
\setlength{\parindent}{2em}              % 设置首行缩进2字符
\XeTeXlinebreaklocale``zh''                                                 % 设置中文换行
\XeTeXlinebreakskip=0pt plus 1pt minus 0.1pt                                % 设置中文换行


%%%%%%%%%%%%%%%%%%%%%%%%%%%%%设置字体%%%%%%%%%%%%%%%%%%%%%%%%%%%%%
\usepackage[CJKchecksingle, CJKnumber]{xeCJK}                 % 用于字体
\setCJKmainfont[BoldFont={Adobe Heiti Std},ItalicFont={Adobe Kaiti Std}]{Adobe Song Std}     % 设置中文/粗体/斜体/主要字体
\setCJKsansfont{Adobe Heiti Std}         % 设置无衬线中文字体
\setCJKmonofont{Adobe Fangsong Std}      % 设置等宽中文字体
\punctstyle{hangmobanjiao}               % 设置行末半角
\setmainfont[Mapping=tex-text]{Times New Roman}         % 设置衬线英文字体
\setsansfont[Mapping=tex-text]{Tahoma}                  % 设置无衬线英文字体
\setmonofont{Courier New}                               % 设置等款英文字体
%%%%%%%%%%%%%%%%%%%%%%%%%%%%%设置图像%%%%%%%%%%%%%%%%%%%%%%%%%%%%%
\usepackage{graphicx}                    % 用于图像编号
\newcommand{\figref}[1]{Fig. \ref{#1}}
\usepackage{tikz}
\usepackage{verbatim}
%%%%%%%%%%%%%%%%%%%%%%%%%%公式%%%%%%%%%%%%%%%%%%%%%%%%%%%%%%%%%%%%%%
\newcommand{\eqsref}[1]{Eq.\eqref{#1}}  %用于公式编号
\newcommand{\myd}{\;\mathrm{d}} %积分的dx
%%%%%%%%%%%%%%%%%%%%%%%%%%%%%设置数学公式%%%%%%%%%%%%%%%%%%%%%%%%%%%%%
\usepackage{amsmath}                     % AMS LaTeX宏包
\usepackage{latexsym}                       %数学符号
\usepackage{amssymb}                       %数学符号
\usepackage{amsfonts}                       %数学符号
\setlength{\mathindent}{0pt}             % 与fleqn配合,使公式靠左显示
%\DeclareMathSizes{12}{12}{10}{10}        % 改变公式大小文本字号、普通公式字号、第一级上下标、第二级上下标
\usepackage{caption}                     % 用来设置字体大小 
\usepackage{pifont}
\usepackage{bbding}
\allowdisplaybreaks
\usepackage{float}
\renewcommand{\today}{\number\year 年 \number\month 月 \number\day 日}     %设置日期为中文方式
%%%%%%%%%%%%%%%%%%%%%%%%%%%%%%水印%%%%%%%%%%%%%%%%%%%%%%%%%%%%%%%%%%%%%%%%%%
%\usepackage{draftwatermark}
%\SetWatermarkText{SJW}%设置水印文字
%\SetWatermarkLightness{0.9}%设置水印亮度
%\SetWatermarkScale{2}%设置水印大小
%\usepackage[contents=SJW, color=gray, opacity=0.8,]{background}
%%%%%%%%%%%%%%%%%%%%%%%%%%%%%%表格%%%%%%%%%%%%%%%%%%%%%%%%%%%%%%%%%%%%%%%%%%
\usepackage{multirow} %数字表格
\usepackage{booktabs} %跨行跨列
\usepackage{colortbl} %彩色表格
\usepackage{paralist} %紧缩式列表

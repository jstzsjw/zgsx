\section{习题2} % (fold)
\label{sec:习题}
\begin{enumerate}
    \item 选择题
        \begin{enumerate}[(1)]
            \item 设$f(x)=\frac{x}{a+e^{bx}}$在$(-\infty,+\infty)$内连续,且$\lim\limits_{x \to -\infty}f(x)=0$,则$(\quad)$。
                \begin{compactenum}[(A)]
                    \item $a<0,b<0$
                    \item $a>0,b>0$
                    \item $a\leq 0,b>0$
                    \item $a \geq 0,b<0$
                \end{compactenum}
            \item 设$f(x)$在$x=a$处可导,则$\left|f(x)\right|$在$x=a$处不可导的充分条件是$(\quad)$。
                \begin{compactenum}[(A)]
                    \item $f(a)=0$且$f^{\prime}(a)=0$
                    \item $f(a)=0$且$f^{\prime}(a) \neq 0$
                    \item $f(a)>0$且$f^{\prime}(a)>0$
                    \item $f(a)<0$且$f^{\prime}(a)<0$
                \end{compactenum}
            \item 下列条件与$f(x)$在$x_0$处可导的定义等价的是$(\quad)$。
                \begin{compactenum}[(A)]
                    \item $\lim\limits_{h \to 0}\frac{f(x_0 +h)-f(x_0 -h)}{2h}$存在
                    \item $\lim\limits_{h \to 0}\frac{f(x_0 +2h)-f(x_0 +h)}{h}$存在
                    \item $\lim\limits_{h \to 0}\frac{f(x_0)-f(x_0 -h)}{h}$存在
                    \item $\lim\limits_{n \to 0}n[f(x_0 + \frac{1}{n})-f(x_0)]$存在
                \end{compactenum}
            \item 设$f(x)$可导,$F(x)=f(x)(1+\left|\sin x\right|)$。则$f(0)=0$是$F(x)$在$x=0$处可导的$(\quad)$。
                \begin{compactenum}[(A)]
                    \item 充要条件
                    \item 充分条件
                    \item 必要条件
                    \item 无关条件
                \end{compactenum}
            \item 设$f(x)$在$(-\delta,\delta)$内有定义,且$\left|f(x)\right|\leq x^2$,则$x=0$是$f(x)$的$(\quad)$。
                \begin{compactenum}[(A)]
                    \item 间断点 
                    \item 连续不可导点
                    \item 可导点且$f^{\prime}(0)=0$
                    \item 可导点且$f^{\prime}(0)\neq 0$
                \end{compactenum}
            \item 设$f(x)=\begin{cases}\frac{2}{3}x^3 & x\leq 1\\
                                        x^2 & x>1
                                    \end{cases}$
                                    则$f^{|prime}$在$x=1$处$(\quad)$。
                \begin{compactenum}[(A)]
                    \item 左、右导数都存在
                    \item 左导数存在,右导数不存在
                    \item 左导数不存在,右导数存在
                    \item 左、右导数都不存在
                \end{compactenum}
        \end{enumerate}
    \item 设$f(x)\begin{cases}x\arctan \frac{1}{x^2} & x\neq 0\\
                                0 & x =0
                            \end{cases}$,则$f^{\prime}(x)$在$x=0$处连续
    \item 设$f(x)$在$(-\frac{\pi}{4},\frac{\pi}{4})$上连续,对任意$x$,$y$满足$f(x+y)=\frac{f(x)+f(y)}{1-f(x)f(y)}$,且$f(x)$在$x=0$处可导,$f^{\prime}(0)=2$。
        \begin{compactenum}[(1)]
            \item 用导数定义求$f^{\prime}(x)$
            \item 求$f(x)$
        \end{compactenum}
    \item $f(x)$是周期为$5$的连续函数,在$x=0$的某个邻域内满足$f(1+\sin x)-3f(1-\sin x)=8x+\alpha(x)$,$\alpha(x)$是当$x\to 0$时比$x$更高阶的无穷小,且$f(x)$在$x=1$处可导,求曲线$y=f(x)$在点$(6,f(6))$处的切线方程。
    \item $f(x)$在$x=0$处满足$f(0)=0$、$f^{\prime}(0)=0$、$f^{\prime\prime}(0)=6$求$\lim\limits_{x \to 0}\frac{f(\sin ^2 x)}{x^4}$。
    \item $f(x)$在$(-\infty,+\infty)$上满足$f(x+1)=2f(x)$,当$0\leq x <1$时$f(x)=x(1-x^2)$。证明$f(x)$在$x=0$处不可导。
    \item 设$f(x)=\begin{cases}\frac{1}{x}-\frac{1}{\ln (1+x)} & x\neq 0,x>-1\\
                                -\frac{1}{2} & x=0
                            \end{cases}$试讨论$f(x)$在$x=0$处的连续、可导性。
    \item 设$f(x)=\max \{\cos x,\left|\frac{2}{\pi}x+1\right|\}$,指出$f(x)$不可导的点,并说明理由。
    \item 令$t=\sqrt{x}$,将方程$4x\frac{d^2 y}{d x^2}+2(1-\sqrt{x})\frac{dy}{dx}-6y=e^{\sqrt[3]{x}}$化为$y$对$t$的微分方程,并求原方程的通解。
    \item 设$f(x)=\begin{cases}\frac{g(x)-e^{-x}}{x} & x \neq 0\\
                                0 & x=0
                            \end{cases}$,其中$g(x)$在$x=0$处存在二阶导数,且$g(0)=1$、$g^{\prime}(0)=-1$,试讨论$f^{\prime}$在$x=0$处的连续性。
    \item 设$\rho =\rho (x)$是抛物线$y=\sqrt{x}$上任一点$M(x,y)\quad(x\geq 1)$处的曲率半径,$s=s(x)$是该曲线上介于点$A(1,1)$与$M$之间的弧长,计算$3\rho \frac{d^2 \rho}{d s^2}-(\frac{d \rho}{d x})^2$的值。(在直角坐标系下曲率的公式$K=\frac{\left|y^{\prime\prime}\right|}{(1+y^{\prime 2})^{\frac{3}{2}}}$)
    \item 从点$\rho_1(1,0)$作$x$轴的垂线,交抛物线$y=x^2$于$Q_1(1,1)$,再从$Q$作抛物线的切线与$x$轴交于$P_2$,过$P_2$作$x$轴的垂线交抛物线于$Q_2$,依次重复得$P_1,Q_1;P_2,Q_2;\dots;P_n,Q_n;\dots$
        \begin{compactenum}[(1)]
            \item 求$\overline{OP_n}$的长度
            \item 求$\sum\limits_{n-1}^{\infty}\overline{Q_n P_n}$的和
        \end{compactenum}
    \item $f(x)\quad n+1$阶可导,$F(x)=\lim\limits_{t \to \infty}t^2 [f(x+\frac{\pi}{t})-f(x)]\sin \frac{x}{t}$,试求$F^{(n)}(x)$。
    \item 试证曲线$\sqrt{x}+\sqrt{y}=\sqrt{a}\quad (a>0)$上任一点的切线在两坐标轴上的截距和为常数。
\end{enumerate}
% section 习题 (end)

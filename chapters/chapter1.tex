\chapter{函数、极限、连续}
\section{考点解析} % (fold)
\label{sec:考点解析}

% section 考点解析 (end)

\section{经典例题} % (fold)
\label{sec:经典例题}

\begin{enumerate}[{例}1.]
    \item   
        \begin{enumerate}[(1)]
            \item $\lim\limits_{x\to 0}(\cos x)^{\frac{1}{\ln{(1+x^2)}}}$
                    \begin{align*}
                    sol&=\lim_{x\to0}\left[1+(\cos x-1)\right]^\frac{1}{\ln{1+x^2}}\\
                        &=\exp{\lim_{x\to0}\frac{\cos x-1}{\ln(1+x^2)}}
                        =\exp{\lim_{x\to0}\frac{-\frac{1}{2}x^2}{x^2}}
                        =e^{-\frac{1}{2}}
                    \end{align*}
            \item $\lim\limits_{x\to+{\infty}}\Big(\sqrt{x+\sqrt{x}}-\sqrt{x-\sqrt{x}}\Big)$
                \begin{align*}
                    sol&=\lim_{x\to+\infty}\frac{2\sqrt{x}}{\sqrt{x+\sqrt{x}}+\sqrt{x-\sqrt{x}}}\\
                       &=\lim_{x\to+\infty}\frac{2}{\sqrt{1+\frac{1}{\sqrt{x}}}+\sqrt{1-\frac{1}{\sqrt{x}}}}
                        =1
                \end{align*}
        \end{enumerate}
    %例1
    \item   
        \begin{enumerate}[(1)]
            \item $\lim\limits_{x\to0}\Big[\frac{a}{x}-(\frac{1}{x^2}-a^2)\ln{(1+ax)}\Big]$, $a$为常数。
                \begin{align*}
                    sol&=\lim_{x\to0}\frac{ax+\ln{(1+ax)}}{x^2}+a^2\lim_{x\to0}\ln{(1+ax)}\\
                       &=\lim_{x\to0}\frac{a-\frac{a}{1+ax}}{2x}+a^2\lim_{x\to0}ax
                        =\lim_{x\to0}\frac{a^2}{2(1+ax)}+0
                        =\frac{1}{2}a^2
                \end{align*}
            \item $\lim\limits_{x\to0}\frac{e^{\sin 2x}-e^{2\sin x}}{x^3}$
                \begin{align*}
                    sol&=\lim_{x\to0}e^{2\sin x}\lim_{x\to0}\frac{e^{\sin 2x-2\sin x}-1}{x^3}\\
                       &=\lim_{x\to0}\frac{\sin 2x+2\sin x}{x^3}\\
                       &=\lim_{x\to0}\frac{2\cos 2x-2\cos x}{3x^2} &or=&\lim_{x\to0}\frac{2\sin x(\cos x-1)}{x^3}\\
                       &=\lim_{x\to0}\frac{-4\sin 2x+2\sin x}{6x}    &=&\lim_{x\to0}-\frac{2\sin x \frac{1}{2}x^2}{x^3}\\
                       &=\lim_{x\to0}\frac{-8\sin 2x+2\cos x}{6}     &=&\lim_{x\to0}-\frac{\sin x}{x}\\
                       &=-1                                          &=&-1
                \end{align*}
        \end{enumerate}
    %例2
    \item  
        \begin{enumerate}[(1)]
            \item $\lim\limits_{x\to0}\frac{(1+x)^{\frac{1}{x}}-e}{x}$
                \begin{align*}
                    sol&=\lim_{x\to0}\frac{e^{\frac{\ln{(1+x)}}{x}}-e}{x}\\
                       &=\lim_{x\to0}\frac{e^{\frac{\ln{(1+x)}}{x}-1}-1}{x}e
                        =\lim_{x\to0}\frac{\frac{\ln{(1+x)}}{x}-1}{x}e\\
                       &=\lim_{x\to0}\frac{\ln{(1+x)}-x}{x^2}e
                        =\lim_{x\to0}\frac{\frac{1}{1+x}-1}{2x}e\\
                       &=\lim_{x\to0}-\frac{1}{2(1+x)}e
                        =-\frac{1}{2}e
                \end{align*}
            \item $\lim\limits_{x\to1}\frac{x-x^x}{1-x+\ln x}$
                \begin{align*}
                    sol&=\lim_{x\to1}\frac{1-x^{(x-1)}}{1-x+\ln x}\lim_{x\to1}x
                        =\lim_{x\to1}\frac{1-e^{(x-1)\ln x}}{1-x+\ln x}\\
                       &=\lim_{x\to1}\frac{-(x-1)\ln x}{1-x+\ln x}
                        =\lim_{x\to1}-\frac{\ln x +\frac{x-1}{x}}{-1+\frac{1}{x}}\\
                       &=\lim_{x\to1}-\frac{\frac{1}{x}+\frac{1}{x^2}}{-\frac{1}{x^2}}
                        =\lim_{x\to1}\frac{x+1}{1}
                        =2
                \end{align*}
            \item $\lim\limits_{n\to\infty}\Big(n\tan\frac{1}{n}\Big)^{n^2}$

                $n\to+\infty$是$x\to+\infty$的特殊情况,将$n$换为$x$,有$\lim\limits_{x\to +\infty}(x\tan \frac{1}{x})^{x^2}$,$x\to +\infty$求极限不方便,倒代换$t= \frac{1}{x}$,
                \begin{align*}
                    sol&\xrightarrow{t=\frac{1}{x}}\lim_{t \to 0^+}(\frac{\tan t}{t})^{\frac{1}{t^2}}
                    =\exp \lim_{t \to 0^+}\frac{\ln(\frac{\tan t}{t})}{t^2}\\
                    &=\exp \lim_{t \to 0^+}\frac{\ln(\frac{\tan t}{t}-1+1)}{t^2}
                    =\exp \lim_{t \to 0^+}\frac{\frac{\tan t}{t}-1}{t^2}\\
                    &=\exp \lim_{t \to 0^+}\frac{\tan t-t}{t^3}
                    =\exp \lim_{t \to 0^+}\frac{\frac{\sin ^2 t}{\cos ^2 t}}{3t^2}\\
                    &=\exp \lim_{t \to 0^+}\frac{t^2}{3t^2}
                    =e^{\frac{1}{3}}
                \end{align*}
            \item $\lim\limits_{x\to0^+}\ln x \ln (1-x)$
                \begin{align*}
                    sol&\xrightarrow{\ln (1-x) \thicksim-x}\lim_{x \to 0^+}-x\ln x\\
                    &=\lim_{x \to 0^+}-\frac{\ln x}{\frac{1}{x}}
                     =\lim_{x \to 0^x}\frac{\frac{1}{x}}{\frac{1}{x^2}}\\
                    &=\lim_{x \to 0^x}x
                     =0
                \end{align*}
        \end{enumerate}
    %例3
    \item 
        \begin{enumerate}[(1)]
            \item $\lim\limits_{x\to+\infty}\Big[x-(1+e^{-x})\ln(1+e^x)\Big]$

                $sol:$
                看$e^x$不顺眼,换
                \begin{enumerate}[$1^\circ$]
                    \item   \begin{align*}
                                &\xrightarrow{t=e^x \to \infty}\lim_{t\to \infty}[\ln t-(1+t^{-1})\ln (1+t)]\\
                                &=\lim_{t \to \infty}[\ln t - \ln (1+t)-\frac{\ln (1+t)}{t}]\\
                                &=\lim_{t \to \infty}[\ln \frac{t}{1+t}-\frac{\ln (1+t)}{t}]\\
                                &=\lim_{t \to \infty}\ln \frac{1}{\frac{1}{t}+1}-\lim_{t \to \infty}\frac{\frac{1}{1+t}}{1}\\
                                &=0-0=0
                            \end{align*}
                    \item   \begin{align*}
                                &\xrightarrow{t=e^{-x} \to 0^+}\lim_{t\to 0^+}[-\ln t-(1+t)\ln (1+t^{-1})]\\
                                &=\lim_{t \to 0^+}[-\ln t - \ln (1+\frac{1}{t})-t \ln (1+\frac{1}{t})]\\
                                &=\lim_{t \to 0^+}[-\ln(t+1)] - \lim_{t \to 0^+} \frac{\ln (1+\frac{1}{t})}{\frac{1}{t}}\\
                                &=\lim_{t \to 0^+}[-\ln(t+1)] - \lim_{t \to 0^+} \frac{\frac{-\frac{1}{t^2}}{1+\frac{1}{t}}}{-\frac{1}{t^2}}\\
                                &=0-0=0
                            \end{align*}
                    \item 或恒等变换
                            \begin{align*}
                                &=\lim_{x \to +\infty}[x-\ln (1+e^x)]-\lim_{x \to +\infty}\frac{\ln(1+e^x)}{e^x}\\
                                &=\lim_{x \to +\infty}[\ln e^x-\ln (1+e^x)]-\lim_{x \to +\infty}\frac{\ln(1+e^x)}{e^x}\\
                                &=\lim_{x \to +\infty}\ln \frac{1+e^x}{e^x}-\lim_{x \to +\infty}\frac{\frac{e^x}{1+e^x}}{e^x}\\
                                &=0-0=0
                            \end{align*}
                \end{enumerate}
            \item $\lim\limits_{x\to0}\bigg(\frac{2+e^{\frac{1}{x}}}{1+e^{\frac{4}{x}}}+\frac{\sin x}{\left|x\right|}\bigg)$
                
                遇绝对值,分左右极限。

                $sol$
                \begin{align*}
                    &\lim_{x \to 0^+}\bigg(\frac{2+e^{\frac{1}{x}}}{1+e^{\frac{4}{x}}}+\frac{\sin x}{\left|x\right|}\bigg)
                    =\lim_{x \to 0^+}\bigg(\frac{2e^{-\frac{4}{x}+e^{-\frac{3}{x}}}}{e^{-\frac{4}{x}+1}}+1\bigg)=1\\
                    &\lim_{x \to 0^-}\bigg(\frac{2+e^{\frac{1}{x}}}{1+e^{\frac{4}{x}}}+\frac{\sin x}{\left|x\right|}\bigg)
                    =1\\
                    &\therefore\mbox{原式}=1
                \end{align*}
            \item $\lim\limits_{x\to0}\frac{3\sin x+x^2\cos\frac{1}{x}}{(1+\cos x)\ln(1+x)}$
                \begin{align*}
                    sol&=\lim_{x \to 0}\frac{3\sin x+ x^2\cos \frac{1}{x}}{2x}
                        =\lim_{x \to 0}\frac{3 \sin x}{2x}+\lim_{x \to 0}\frac{x\cos\frac{1}{x}}{2}\\
                        &=\frac{3}{2}
                \end{align*}
        \end{enumerate}
    %例4
    \item  
        \begin{enumerate}[(1)]
            \item $\lim\limits_{x\to{+\infty}}\bigg(\frac{\pi}{2}-\arctan x\bigg)^{\frac{1}{\ln x}}$

                $sol=\exp \lim\limits_{x \to +\infty}\frac{\ln(\frac{\pi}{2}-\arctan x)}{\ln x}=\exp \lim\limits_{x \to +\infty}\frac{-x}{(1+x^2)(\frac{\pi}{2}-\arctan x)}$
                \begin{enumerate}[$1^\circ$]
                    \item 
                        \begin{align*}
                            &=\exp \lim_{x \to +\infty} \frac{-\frac{1}{x}}{(\frac{1}{x^2}+1)(\frac{\pi}{2}-\arctan x)}\\
                            &=\exp \lim_{x \to +\infty} \frac{-\frac{x}{1+x^2}}{\frac{\pi}{2}-\arctan x}\\
                            &=\exp \lim_{x \to +\infty} \frac{-\frac{1-x^2}{(1+x^2)^2}}{-\frac{1}{1+x^2}}\\
                            &=\exp \lim_{x \to +\infty} \frac{1-x^2}{1+x^2}=e^{-1}
                        \end{align*}
                    \item
                        \begin{align*}
                            &=\exp \lim_{x \to +\infty} \frac{-\frac{1}{x}}{(1+\frac{1}{x^2})(\frac{\pi}{x}-\arctan x)}\\
                            &=\exp \lim_{x \to +\infty} \frac{-\frac{1}{x}}{\frac{\pi}{2}-\arctan x}\\
                            &=\exp \lim_{x \to +\infty} \frac{\frac{1}{x^2}}{-\frac{1}{1+x^2}}=e^{-1}
                        \end{align*}
                    \item
                        \begin{align*}
                            &=\exp \lim_{x \to +\infty} \frac{-1}{2x(\frac{\pi}{2}-\arctan x)-1}\\
                            &\xrightarrow{\mbox{先求}}\lim_{x \to +\infty} x(\frac{\pi}{2}-\arctan x)\\
                            &=\lim_{x \to +\infty} \frac{\frac{\pi}{2}-\arctan x}{\frac{1}{x}}\\
                            &=\lim_{x \to +\infty} \frac{-\frac{1}{1+x^2}}{-\frac{1}{x^2}}=1\\
                            &\therefore \exp \lim_{x \to +\infty}\frac{-1}{2-1}=e^{-1}
                        \end{align*}
                    \end{enumerate}
            \item $\lim\limits_{x\to 0^+}\frac{1-\cos x}{\sqrt{1-x}-\cos\sqrt{x}}$
                \begin{align*}
                    sol&=\lim_{x \to 0^+} \frac{\sqrt{1-x}\cos x}{1-x-\cos ^2 \sqrt{x}}\frac{1-\cos ^2 x}{1+\cos ^2 x}\\
                       &=\lim_{x \to 0^+}\frac{\sqrt{1x}+\cos \sqrt{x}}{1+\cos x}\frac{\sin ^2 x}{1-x-\cos ^2 \sqrt{x}}\\
                       &=\lim_{x \to 0^+}\frac{2}{2} \frac{x^2}{\sin ^2 \sqrt{x}-x}\\
                       &\xrightarrow{t=\sqrt{x}}\lim_{x \to 0^+}\frac{t^4}{\sin ^2 t-t^2}
                       =\lim_{x \to 0^+}\frac{2t^3}{\sin t \cos t-t}\\
                       &=\lim_{x \to 0^+}\frac{6t^2}{\cos ^2 t-\sin ^2 t -1}
                        =\lim_{x \to 0^+}\frac{3t^2}{-\sin ^2 t}\\
                       &=-3
                \end{align*}
        \end{enumerate}
    %例5
    \item 已知$\lim\limits_{x\to\infty}\Big[(x^5+6x^4+2)^\alpha-x\Big]=\beta$, 求$\alpha$,$\beta$的值。
        \begin{align*}
            sol:&\because \lim\limits_{x\to\infty}\Big[(x^5+6x^4+2)^\alpha-x\Big]=\beta\\
            &\therefore \lim_{x \to +\infty}[(x^5+7x^4+2)^{\alpha}-x]=\lim_{x \to +\infty}x\bigg[\frac{(x^5+7x^4+2)^{\alpha}}{x}-1\bigg]=0\\
            &\mbox{即}\lim_{x \to +\infty}\bigg(\frac{x^5+7x^4+2}{x^{\frac{1}{\alpha}}}\bigg)^\alpha =1\\
            &\therefore \frac{1}{\alpha}=5 \Rightarrow \alpha=5
        \end{align*}
            又 $\because \beta=\lim\limits_{x \to +\infty}(\sqrt[5]{x^5+7x^4+2}-x)$
        \begin{enumerate}[$1^\circ$]
            \item 
                \begin{align*}
                    &=\lim_{x \to +\infty}x(\sqrt[5]{x^5+7x^4+2}-1)\\
                    &=\lim_{x \to +\infty}x\frac{1}{5}(\frac{7}{x}+\frac{2}{x^5})=\frac{7}{5}
                \end{align*}
            \item 
                \begin{align*}
                    &\xrightarrow{t=\frac{1}{x}\to 0^+}\lim_{t \to 0^+}\bigg(\sqrt[5]{\frac{1}{t^5}+\frac{7}{t^4}+2}-\frac{1}{t}\bigg)\\
                    &=\lim_{t \to 0^+} \frac{1}{t}(\sqrt[5]{1+7t+2t^5}-1)\\
                    &=\lim_{t \to 0^+} \frac{1}{t}\cdot\frac{1}{5}(7t+2t^5)=\frac{7}{5}
                \end{align*}
        \end{enumerate}

    %例6
    \item 设$f(x)$在$(0,+\infty)$内可导,$f(x)>0$,$\lim\limits_{x\to{+\infty}}f(x)=1$且$\lim\limits_{h\to0}\bigg[\frac{f(x+hx)}{f(x)}\bigg]^\frac{1}{h}=e^\frac{1}{x}$,求$f(x)$。
        \begin{align*}
            sol:&\because \lim_{h \to 0}\bigg[\frac{f(x+hx)}{f(x)}\bigg]^{\frac{1}{h}}=e^{\frac{1}{x}}\\
            &\therefore \lim_{h \to 0}\frac{\ln \frac{f(x+hx)}{f(x)}}{h}=\frac{1}{x}
        \end{align*}
        \begin{enumerate}[$1^\circ$]
            \item ``$+1-1$''法
                \begin{align*}
                    \lim_{h \to 0}\frac{\ln(\frac{f(x+hx)-f(x)}{f(x)}+1)}{h}&=\frac{1}{x}\\
                    \lim_{h \to 0}\frac{f(x+hx)-f(x)}{hf(x)}&=\frac{1}{x}\\
                    \lim_{h \to 0}\frac{f(x+hx)-f(x)}{hx}&=\frac{1}{x^2}f(x)\\
                    f^{\prime}(x)&=\frac{1}{x^2}f(x)\\
                    \therefore f(x)&=e^{-\frac{1}{x}}
                \end{align*}
            \item 
                \begin{align*}
                    \lim_{h \to 0} \frac{\ln f(x+hx)-\ln f(x)}{h}&=\frac{1}{x}\\
                    \lim_{h \to 0} \frac{\ln f(x+hx)-\ln f(x)}{hx}\cdot x&=\frac{1}{x}\\
                    x\cdot[\ln f(x)]^{\prime}&=\frac{1}{x}\\
                    \therefore (\ln f(x))^{\prime}&=\frac{1}{x}\\
                    \ln f(x)&=\frac{1}{x}\\
                    f(x)&=e^{-\frac{1}{x}}
                \end{align*}
            \item 泰勒展开法
                \begin{align*}
                    f(x)=f(x_0)+f^{\prime}(x_0)(x-x_0)+o(x-x_0)\\
                    \therefore f(x+\Delta x)=f(x)+f^{\prime}(x)\Delta x+o(\Delta x)\\
                    \therefore \ln f(x+hx)=\ln f(x)+[\ln f(x)]^{\prime}hx+o(hx)\\
                    \mbox{则}\lim_{h \to 0}\frac{\ln f(x+hx)-\ln f(x)}{h}&=\frac{1}{x}\\
                    \lim_{h \to 0}\frac{[\ln f(x)]^{\prime}hx+o(hx)}{h}&=\frac{1}{x}\\
                    [\ln f(x)]^{\prime}&=\frac{1}{x^2}\\
                    \therefore f(x)&=e^{-\frac{1}{x}}
                \end{align*}
        \end{enumerate}
        %例7
    \item 设$f(x)$在$x=0$处可导,$f(0)\neq0$,$f^\prime(0)\neq0$,当$h\to0$时,$af(h)+bf(2h)-f(0)$是比$h$更高阶的无穷小,求$a$,$b$的值。
        \begin{align*}
            sol:&\because \lim_{h \to 0} af(h)+bf(2h)-f(0)=0\\
            &\mbox{并且}f(x)\mbox{在}x=0\mbox{可导,}\rightarrow\mbox{连续,代入}\\
            &af(0)+bf(0)-f(0)=0\\
            &(a+b-1)f(0)=0\\
            &a+b=1\\
            &\mbox{回代}\\
            &\lim_{h \to 0}\frac{af(h)+bf(2h)-(a+b)f(0)}{h}\\
            &=\lim_{h \to 0}a\frac{f(h)-f(0)}{h}+\lim_{h \to 0}2b\frac{f(2h)-f(0)}{2h}\\
            &=af^{\prime}(0)+2bf^{\prime}(0)=0\\
            &\therefore
            \begin{cases}
                a+b=1\\
                a+2b=0
            \end{cases}
            \Rightarrow
            \begin{cases}
                a=2\\
                b=-1
            \end{cases}
        \end{align*}
    %例8
    \item 设$f(x)$为连续函数,$f(0)\neq0$,求极限$\lim\limits_{x\to0}\frac{\int_0^x (x-t)f(t)dt}{x\int_0^xf(t)dt}$。

        $sol:$
        \begin{align*}
            &\lim_{x \to 0}\frac{xf(x)+\int_0^x f(t)dt-xf(x)}{xf(x)+\int_0^x f(t)dt}\\
            &=\lim_{x \to 0}\frac{\int_0^x f(t)dt}{xf(x)+\int_0^x f(t)dt}
        \end{align*}
        \begin{enumerate}[$1^\circ$]
            \item 
                \begin{align*}
                    \mbox{原式}&=\lim_{x \to 0}\frac{\frac{\int_0^x f(t)dt}{x}}{f(x)+\frac{\int_0^x f(t) dt}{x}}\\
                    \mbox{先计算}&\lim_{x \to 0}\frac{\int_0^x f(t) dt}{x}\\
                    &=\lim_{x \to 0}f(x)\\
                    &=f(0)\\
                    \therefore\mbox{原式}&=\lim_{x \to 0}\frac{f(0)}{f(x)+f(0)}\\
                    &=\frac{1}{2}
                \end{align*}
            \item 泰勒展开
                \begin{align*}
                    \mbox{设}F(x)&=\int_0^x f(t)dt\\
                    F(x)&=F(0)+F^{\prime}(0)x+o(x)\\
                    &=0+f(x)x+o(x)\\
                    \therefore\mbox{原式}&=\lim_{x \to 0}\frac{F(x)}{xf(x)+F(x)}\\
                    &=\lim_{x \to 0}\frac{xf(x)+o(x)}{xf(x)+xf(x)+o(x)}\\
                    &=\lim_{x \to 0}\frac{f(x)+\frac{o(x)}{x}}{2f(x)+\frac{o(x)}{x}}\\
                    &=\frac{1}{2}
                \end{align*}
        \end{enumerate}
    %例9
    \item 当$x\to0$时,$1-\cos x\cos 2x\cos 3x$与$ax^n$为等价无穷小,求常数$a$与$n$的值。
        \begin{align*}
            sol:&\mbox{泰勒展开}\\
                &\cos x=1-\frac{1}{2}x^2+o(x^2)\\
                &\therefore 1-\cos x \cos 2x \cos 3x\\
                &=1-[1-\frac{1}{2}x^2+o(x^2)][1-\frac{1}{2}(2x)^2+o(x^2)][1-\frac{1}{2}(3x)^2+o(x^2)]\\
                &=(\frac{1}{2}+\frac{4}{2}+\frac{9}{2})x^2+o(x^2)\\
                &=7x^2+o(x^2)\\
                &\because\mbox{与}ax^n\mbox{等价无穷小}\\
                &\therefore n=2,a=7
        \end{align*}
    %例10
    \item 当$x\to0$时,$x^2+\ln(1+x)\ln(1-x)$与$ax^n$为等价无穷小,求常数$a$与$n$的值。
        \begin{align*}
            sol:&\mbox{泰勒展开}\\
                &\ln (1+x)=x-\frac{1}{2}x^2+\frac{1}{3}x^3+o(x^3)\\
                &\therefore x^2+\ln(1+x)\ln(1-x)\\
                &=x^2+[x-\frac{1}{2}x^2+\frac{1}{3}x^3+o(x^3)][-x-\frac{1}{2}(-x)^2+\frac{1}{3}(-x)^3-o(x^3)]\\
                &=x^2-[x-\frac{1}{2}x^2+\frac{1}{3}x^3+o(x^3)][x+\frac{1}{2}x^2+\frac{1}{3}x^3+o(x^3)]\\
                &=(-\frac{1}{3}-\frac{1}{3}+\frac{1}{4})x^4+o(x^4)\\
                &=-\frac{5}{12}x^4+o(x^4)\\
                &\because\mbox{与}ax^n\mbox{等价无穷小}\\
                &\therefore n=4,a=-\frac{5}{12}
        \end{align*}
    %例11
    \item 已知$\lim\limits_{x\to0}\frac{a\tan x+b(1-\cos x)}{c\ln(1-2x)+d(1-e^{-x^2})}=2$,$(a^2+c^2\neq0)$,则$(\quad)\\ A.b=4d\quad B.b=-4d\quad C.a=4c\quad D.a=-4c$

        $sol:$
        \begin{enumerate}[$1^\circ$]
            \item 洛必达
                \begin{align*}
                    \lim_{x \to 0}\frac{a\frac{1}{\cos ^2 x}+b\sin x}{c\frac{-2}{1-2x}+d2xe^{-x^2}}=\frac{a}{-2c}=2\Rightarrow a=-4c
                \end{align*}
            \item
                \begin{align*}
                    \lim_{x \to 0}\frac{a\frac{\tan x}{x}+b\frac{\ln(1-\cos x)}{x}}{c\frac{\ln(1-2x)}{x}+d\frac{1-e^{-x^2}}{x}}=\frac{a}{-2c}=2\Rightarrow a=-4c
                \end{align*}
        \end{enumerate}
    %例12
    \item 设$f(x)$对一切正数$x_1$,$x_2$有$f(x_1,x_2)=f(x_1)+f(x_2)$,且$f(x)$在$x=1$处连续,证明$f(x)$在$(0,+\infty)$连续。
        \begin{align*}
            sol:&\mbox{取}x_1 = x_2 =1\Rightarrow f(1)=f(1)+f(1)\Rightarrow f(1)=0\\
                &\lim_{\Delta x\to 0}f(x+\Delta x)=\lim_{\Delta x\to 0}f\bigg[x\Big(1+\frac{\Delta x}{x}\Big)\bigg]\\
                &=\lim_{\Delta x}\bigg[f(x)+f\Big(1+\frac{\Delta x}{x}\Big)\bigg]\\
                &\because f(x)\mbox{在}x=1\mbox{处连续}\\
                &\therefore \lim_{\Delta x\to 0}f\Big(1+\frac{\Delta x}{x}\Big)=f(1)=0\\
                &\therefore \lim_{\Delta x\to 0}f(x+\Delta x)=\lim_{\Delta x\to 0}[f(x)+0]=f(x)\\
                &\therefore f(x)\mbox{在}(0,+\infty)\mbox{连续}
        \end{align*}
    %例13
    \item 设$f(x)$在$[0,+\infty)$上可导,$f(0)<0$,$f^{\prime}(x)\geq k>0$,($k$为常数),证明方程$f(x)=0$在$(0,+\infty)$上有唯一根。
        \begin{align*}
            sol:&\mbox{构造}F(x)=f(x)-f(0)-kx\\
                &F^{\prime}(x)=f^{\prime}(x)-k\geq 0\\
                &\mbox{又}\because F(0)=0\\
                &\therefore F(x)\geq F(0)=0\\
                &\therefore f(x)\geq f(0)+kx\\
                &\because\mbox{当}f(0)+kx_0=0\Rightarrow x_0=-\frac{f(0)}{k}>0\mbox{时}f(-\frac{f(0)}{k})\geq 0\\
                &\mbox{且}f(0)<0,f^{\prime}(x)\geq k>0\\
                &\therefore f(x)=0\mbox{在}(0,+\infty)\mbox{上有唯一根}
        \end{align*}
    %例14
    \item 设$f(x)$在$[a,+\infty)$上二阶可导,$f(a)<0$,$f^\prime(a)>0$当$x>a$时,$f^{\prime\prime}(x)>0$,证明方程$f(x)$在$(a,+\infty)$上存在唯一根。
        \begin{align*}
            sol:&\because f^{\prime\prime}(x)>0\\
                &\therefore \mbox{当}x\in (a,+\infty)\mbox{时}f^{\prime}(x)>f^{\prime}(a)\\
                &\mbox{构造}F(x)=f(x)-f(a)-f^{\prime}(a)x\\
                &F^{\prime}(x)=f^{\prime}(x)-f^{\prime}(a)>0\quad(x\in(a,+\infty))\\
                &\therefore f(x)-f(a)-f^{\prime}(a)x>-f^{\prime}(a)a\\
                &f(x)>f(a)+f^{\prime}(a)x=f^{\prime}(a)a\\
                &\mbox{令}f(a)+f^{\prime}(a)x_0-f^{\prime}(a)a=0\Rightarrow x_0=a-\frac{f(a)}{f^{\prime}(a)}>a\\
                &\therefore f(x_0)>0\\
                &\mbox{又}\because f(a)<0,f^{\prime}(x)>f^{\prime}(a)>0\\
                &\therefore f(x)=0\mbox{在}(a,+\infty)\mbox{上存在唯一跟}
        \end{align*}
    %例15
    \item 设$x_1>0$,当$n\geq1$时,$x_{n+1}=\frac{1}{2}(x_n+\frac{a}{x_n})$,$a$为正常数,证明数列$\{x_n\}$存在极限并求其极限。
        \begin{align*}
            sol:&x_{n+1}-x_n=\frac{1}{2}\Big(x_n+\frac{a}x_n\Big)-x_n=\frac{a-x_n ^2}{2x_n}\\
                &x_{n+1}=\frac{1}{2}\Big(x_n+\frac{a}{x_n}\Big)\geq \sqrt{a}\\
                &\therefore x_{n+1}-x_n\leq 0\\
                &\therefore n\geq 2\mbox{时}x_n\mbox{单调递减,有下界为}\sqrt{a}\Rightarrow \lim_{n \to \infty}x_n\mbox{存在}\\
                &\mbox{设}\lim_{n \to \infty}x_n=A\mbox{,在}x_{n+1}=\frac{1}{2}\Big(x_n+\frac{a}{x_n}\Big)\mbox{两边令}n\to \infty\\
                &\mbox{则有}A=\frac{1}{2}\Big(A+\frac{a}{A}\Big)\Rightarrow A=\sqrt{a}
        \end{align*}
    %例16
    \item 设$a>0$,$x_1>\sqrt{a}$,当$n\geq1$时,$x_{n+1}=\sqrt{a+x_n}$,证明$\{x_n\}$存在极限并求出其极限。
        \begin{align*}
            sol:&x_{n+a}^2=a+x_n\\
                &\therefore x_{n+1}^2-x_n^2=x_n-x_{n-1}\\
                &\because x_3^2-x_2^2=x_2-x_1=\sqrt{a+\sqrt{a}}-\sqrt{a}>0\\
                &\therefore x_3>x_2\\
                &x_{n+1}=\frac{a}{x_{n+a}}+\frac{x_n}{x_{n+1}}\leq \frac{a}{x_1}+1=\sqrt{a}+1\\
                &\mbox{显然}x_{n+1}>x_n\mbox{,}\{x_n\}\mbox{单调增加,有上界为}\sqrt{a}+1\Rightarrow\lim_{n \to \infty}x_n\mbox{存在}\\
                &\mbox{设}\lim_{n \to \infty}x_n=A\mbox{,在}x_{n+1}^2=a+x_n\mbox{两边令}n\to \infty\\
                &\mbox{则有}A^2=a+A\Rightarrow A=\frac{1}{2}\textpm\sqrt{\frac{1}{4}+a}\\
                &\mbox{又}\because A\geq \sqrt{a}>0\quad\therefore A=\frac{1}{2}+\sqrt{\frac{1}{4}+a}
        \end{align*}
    %例17
    \item 设$x_1>0$,当$n\geq1$时,$x_{n+1}=\sqrt{a x_n}$,证明$\{x_n\}$存在极限并求出其极限。
    %例18
    \item 设$x_1=2$,当$n\geq1$时,$x_{n+1}=1+\frac{x_n}{1+x_n}$,证明$\{x_n\}$存在极限并求出其极限。
    %例19
    \item 设$0<x_1<3$,当$n\geq1$时,$x_{n+1}=\sqrt{x_n(3-x_n)}$,证明$\{x_n\}$存在极限并求出其极限。
    %例20
    \item 设$0<a<1$,$x_1 =\frac{a}{2}$,当$n\geq1$时,$x_{n+1}=\frac{a+x_n^2}{2}$,证明$\{x_n\}$存在极限并求出其极限。
    %例21
    \item 设$a_0\geq b_0>0$,当$n\geq 0$时,$a_{n+1}=\frac{a_n+b_n}{2}$,$b_{n+1}=\frac{2 a_n b_n}{a_n + b_n}$,证明数列$\{a_n\}$与$\{b_n\}$的极限均存在,并求$\lim\limits_{n\to \infty}a_n$,$\lim\limits_{n \to \infty}b_n$。
    %例22
    \item 设$x_0>0$,$y_0>0$,当$n\geq 0$时,$x_{n+1}=\sqrt{x_n y_n}$,$y_{n+1}=\frac{x_n + y_n}{2}$,证明数列$\{x_n\}$与$\{y_n\}$的极限均存在,且$\lim\limits_{n\to \infty}x_n=\lim\limits_{n \to \infty}y_n$。
    %例23
    \item 设$x_1=a$,$x_2=b$,$a<b$,当$n\geq 2$时,$x_{n+1}=\frac{x_n +x_{n-1}}{2}$,求
        \begin{compactenum}[(1)]
                \item $y_n=x_n-x_{n-1}$,$n\geq 2$;
                \item $\sum\limits_{k=2}^{n}y_k$;
                \item $\lim\limits_{n \to \infty}x_n$。
            \end{compactenum}
    %例24
    \item 设$x_1=1$,$x_2=2$,当$n\geq 2$时,$x_{n+1}=\sqrt{x_n x_{n+1}}$,求
        \begin{compactenum}[(1)]
                \item $y_n=\ln x_n - \ln x_{n-1}$,$n\geq 2$;
                \item $\sum\limits_{k=2}^{n}y_k$;
                \item $\lim\limits_{n \to \infty}x_n$。
            \end{compactenum}
    %例25
    \item 设$n$为正整数,且$n \pi \leq x < (n+1) \pi$,
            \begin{compactenum}[(1)]
                \item 证明$2n\leq \int_{0}^{x}\left|\sin t\right| dt < 2(n+1)$;
                \item 求$\lim\limits_{x\to +\infty}\frac{\int_{0}^{x}\left|\sin t\right|dt}{x}$。
            \end{compactenum}
    %例26
    \item 当$0<x<1$时,证明$\sin \frac{\pi x}{2} > x$;又设$0<x_1<1$,当$n\geq 1$时,$x_{n+1}=\sin\frac{\pi x_n}{2}$,证明数列$\{x_n\}$存在极限并求出其极限。
    %例27
    \item 证明方程$x^n+x^{n+1}+\dots+x^2+x=1$,$(n\geq 2)$在$(0,1)$上存在唯一根,并将此根记为$x_n$,证明数列$\{x_n\}$存在极限并求出其极限。
\end{enumerate}
% section 经典例题 (end)

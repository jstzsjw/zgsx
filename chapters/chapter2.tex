\chapter{导数与微分}
\section{考点解析} % (fold)
\label{sec:考点解析}

% section 考点解析 (end)

\section{经典例题} % (fold)
\label{sec:经典例题}

\begin{enumerate}[{例}1.]
    \item 设$f(x)$在$x_0$处二阶可导,求$\lim\limits_{h \to 0}\frac{f(x_0+h)+f(x_0-h)-2f(x_0)}{h^2}$。
        %例题1
    \item 设$f(0)=0$,$f^{\prime}(0)=1$,$f^{\prime\prime}(0)=2$,求$\lim\limits_{x \to 0}\frac{f(x)-x}{x^2}$。
        %例题2
    \item 设$f(x)=\begin{cases}\frac{\sin x}{x} & x \neq 0\\
                               1 & x = 0
                           \end{cases}$,求$f^{\prime}(0)$,$f^{\prime\prime}(0)$。
        %例题3
    \item 设$f(x)=\begin{cases}\frac{1}{x}-\frac{1}{e^x -1} & x \neq 0\\
                                                    \frac{1}{2} & x = 0
                    \end{cases}$,试讨论$f(x)$在$x=0$处的连续性、可导性。
        %例题4
    \item $f(x)$在$x=0$处二阶可导,$f(0)=0$,$g(x)=\begin{cases}\frac{f(x)}{x} & x \neq 0\\
                                                                f^{\prime}(0) & x = 0
                                                            \end{cases}$,证明$g^{\prime}(x)$在$x=0$处连续。
        %例题5
    \item 设$f(x)$在$(-\infty,+\infty)$上连续,$\lim\limits_{x \to 0}\frac{f(x)}{x}=A$,$A$为常数,$\varphi(x)=\int_0^1 f(xt)dt$,试讨论$\varphi^{\prime}(x)$的连续性。
        %例题6
    \item 设$f(x)$在$(-\infty,+\infty)$上有定义,对任意$x$、$y$有$f(x+y)=e^xf(y)+e^yf(x)$,且$f(x)$在$x=0$处可导,$f^{\prime}(0)=2$,证明$f(x)$在任一点处可导,并求$f(x)$。
        %例题7
    \item 设$f(x)$在$(-\infty,+\infty)$上连续,$f(1)=3$,对任意$x>0$,$y>0$有$\int_1^{xy}f(x)dt=x\int_1^y f(t)dt+y\int_1^x f(t)dt$,求$f(x)$。
        %例题8
    \item 设$f(x)=\begin{cases}x^3 \sin \frac{1}{x} & x\neq 0\\
                                0 & x=0
                \end{cases}$,求使得$f^{(n)}(0)$存在的最大$n$。
        %例题9
    \item 设$f(x)=3x^3 + x^2 \left|x\right|$,求使得$f^{(n)}(0)$存在的最大$n$。
        %例题10
    \item 设$y=(2x+3)^3(3x+2)^2$,求$y^{(5)}$,$y^{(6)}$。
        %例题11
    \item 设$y=\frac{x^4-x^3+2x^2-3x}{x-1}$,求$y^{(5)}$。
        %例题12
    \item 设$y=y(x)$由方程$y=\cos x +xe^y$所确定,求$y^{\prime}(0)$,$y^{\prime\prime}(0)$。
        %例题13
    \item 设$y=y(x)$由方程$2y^3-2y^2+2xy-x^2 =1$所确定,求$f(x)$的驻点,并判别该驻点是否为极值点。
        %例题14
    \item 通过变换$x=\sin t$化简方程$(1-x^2)\frac{d^2y}{dx^2}-x\frac{dy}{dx}+a^2 y=0$,并求原方程的通解。
        %例题15
    \item 通过变换$x=\frac{u}{\cos x}$化简方程$\cos x \frac{d^2 y}{d x^2}-2\sin x \frac{dy}{dx}+3y \cos x=e^x$,并求原方程的通解。
        %例题16
    \item 设$f(x)=y(x)$满足方程$y^{\prime\prime}+(x+e^{2y})y^{\prime 3}=0$,$y^{\prime}\neq 0$,试将该方程化为$y=y(x)$的反函数$x=x(y)$满足的微分方程,并求原方程的通解。
        %例题17
\end{enumerate}
% section 经典例题 (end)

\chapter{导数与微分}
\section{考点解析} % (fold)
\label{sec:考点解析}

% section 考点解析 (end)

\section{经典例题} % (fold)
\label{sec:经典例题}

\begin{enumerate}[{例}1.]
    \item 设$f(x)$在$x_0$处二阶可导,求$\lim\limits_{h \to 0}\frac{f(x_0+h)+f(x_0-h)-2f(x_0)}{h^2}$。

        $sol:$
        \begin{enumerate}[$1^\circ$]
            \item
                \begin{align*}
                    &\because f(x)\mbox{在}x_0\mbox{处二阶可导}\\
                    &\therefore f(x)\mbox{一阶导数在}x_0\mbox{的某个邻域内有定义}\\
                    &\therefore \lim_{h \to 0}\frac{f(x_0+h)+f(x_0-h)-2f(x_0)}{h^2}\\
                    &=\lim_{h \to 0}\frac{f^{\prime}(x_0+h)-f^{\prime}(x_0-h)}{2h}\\
                    &=\frac{1}{2}\bigg[\lim_{h \to 0}\frac{f^{\prime}(x_0+h)-f^{\prime}(x_0)}{h}+\lim_{h \to 0}\frac{f^{\prime}(x_0-h)-f^{\prime}(x_0)}{-h}\bigg]\\
                    &=\frac{1}{2}\cdot 2 f^{\prime\prime}(x_0)=f^{\prime\prime}(x_0)
                \end{align*}
            \item 泰勒展开
                \begin{align*}
                    &f(x_0+h)=f(x_0)+f^{\prime}(x_0)h+\frac{1}{2}f^{\prime\prime}(x_0)h^2+o(h^2)\\
                    &\therefore\lim_{h \to 0}\frac{f(x_0+h)+f(x_0-h)-2f(x_0)}{h^2}\\
                    &=\lim_{h \to 0}\frac{f(x_0)+f^{\prime}(x_0)h+\frac{1}{2}f^{\prime\prime}(x_0)h^2+o(h^2)}{h^2}\\
                    &+\lim_{h \to 0}\frac{f(x_0)-f^{\prime}(x_0)h+\frac{1}{2}f^{\prime\prime}(x_0)h^2+o(h^2)-2f(x_0)}{h^2}\\
                    &=\lim{h \to 0}\frac{f^{\prime\prime}(x_0)h^2+o(h^2)}{h^2}\\
                    &=f^{\prime\prime}(x_0)
                \end{align*}
        \end{enumerate}
        %例题1
    \item 设$f(0)=0$,$f^{\prime}(0)=1$,$f^{\prime\prime}(0)=2$,求$\lim\limits_{x \to 0}\frac{f(x)-x}{x^2}$。

        $sol:$
        \begin{enumerate}[$1^\circ$]
            \item
                \begin{align*}
                    &\lim_{x \to 0}\frac{f(x)-x}{x^2}= \lim_{x\to 0}\frac{f^{\prime}(x)-1}{2x}= \lim_{x\to 0}\frac{f^{\prime\prime}}{2}= 1
                \end{align*}
            \item
                \begin{align*}
                    &f(x+0)=f(0)+f^{\prime}(0)x+\frac{1}{2}f^{\prime\prime}(0)x^2+o(x^2)=x+x^2+o(x^2)\\
                    &\lim_{x \to 0}\frac{f(x)-x}{x^2}=\lim_{x\to 0}\frac{x+x^2+o(x^2)-x}{x^2}=\lim_{x\to 0}\frac{x^2+o(x^2)}{x^2}=1
                \end{align*}
        \end{enumerate}
        %例题2
    \item 设$f(x)=\begin{cases}\frac{\sin x}{x} & x \neq 0\\
                               1 & x = 0
                           \end{cases}$,求$f^{\prime}(0)$,$f^{\prime\prime}(0)$。
        \begin{align*}
            sol:&f^{\prime}(0)=\lim_{x\to 0}\frac{\frac{\sin x}{x}-1}{x}=\lim_{x \to 0}\frac{\sin x-x}{x^2}=\lim_{x\to 0}\frac{\cos x-1}{2x}\\
                &=\lim_{x\to 0}\frac{\frac{1}{2}x^2}{2x}=0\quad or\quad=\lim_{x\to 0}\frac{-\sin x}{2}=0\\
                &x\neq 0\mbox{时}f^{\prime}(x)=\frac{x\cos x-\sin x}{x^2}\\
                &\therefore f^{\prime\prime}(0)=\lim_{x\to 0}\frac{f^{\prime}(x)-f^{\prime}(0)}{x}\\
                &=\lim_{x\to 0}\frac{x\cos x-\sin x}{x^3}\\
                &=\lim_{x\to 0}\frac{\cos x-x\sin x-\cos x}{3x^2}\\
                &=\lim_{x\to 0}\frac{-\sin x}{3x}=-\frac{1}{3}
        \end{align*}
        %例题3
    \item 设$f(x)=\begin{cases}\frac{1}{x}-\frac{1}{e^x -1} & x \neq 0\\
                                                    \frac{1}{2} & x = 0
                    \end{cases}$,试讨论$f(x)$在$x=0$处的连续性、可导性。
                    
            $sol:$
            \begin{enumerate}[(1)]
                \item 连续性
                    \begin{align*}
                        &\lim_{x\to 0}f(x)=\lim_{x\to 0}\frac{1}{x}-\frac{1}{e^x-1}\\
                        &=\lim_{x\to 0}\frac{e^x-1-x}{x(e^x-1)}=\lim_{x\to 0}\frac{e^x-1-x}{x^2}=\lim_{x\to 0}\frac{e^x-1}{2x}=\lim_{x\to 0}\frac{e^x}{2}\\
                        &=\frac{1}{2}=f(0)\\
                        &\therefore\mbox{连续}
                    \end{align*}
                \item 可导性
                    \begin{align*}
                        &f^{\prime}(0)=\lim_{x\to 0}\frac{f(x)-f(0)}{x}\\
                        &=\lim_{x\to 0}\frac{\frac{1}{x}-\frac{1}{e^x-1}-\frac{1}{2}}{x}=\lim_{x\to 0}\frac{(2-x)(e^x-1)-2x}{2x^2(e^x-1)}\\
                        &=\lim_{x\to 0}\frac{(2-x)e^x-2-x}{2x^3}=\lim_{x\to 0}\frac{-e^x+(2-x)e^x-1}{6x^2}\\
                        &=\lim_{x\to 0}\frac{-e^x+(1-x)e^x}{12x}=\lim_{x\to 0}\frac{-xe^x}{12x}=-\frac{1}{12}\\
                        &\therefore\mbox{可导}
                    \end{align*}
            \end{enumerate}
        %例题4
    \item $f(x)$在$x=0$处二阶可导,$f(0)=0$,$g(x)=\begin{cases}\frac{f(x)}{x} & x \neq 0\\
                                                                f^{\prime}(0) & x = 0
                                                            \end{cases}$,证明$g^{\prime}(x)$在$x=0$处连续。
            \begin{align*}
                sol:&g^{\prime}(0)=\lim_{x\to 0}\frac{\frac{f(x)}{x}-f^{\prime}(0)}{x}\\
                    &=\lim_{x\to 0}\frac{f(x)-xf^{\prime}(0)}{x^2}=\lim_{x\to 0}\frac{f^{\prime}(x)-f^{\prime}(0)-xf^{\prime\prime}(0)}{2x}=\frac{1}{2}f^{\prime\prime}\\
                    &x\neq 0\mbox{时}\quad g^{\prime}(x)\frac{f^{\prime}(x)-f(x)}{x^2}\\
                    &\lim_{x\to 0}g^{\prime}(x)=\lim_{x\to 0}\frac{f^{\prime}(x)x-f(x)}{x^2}=\lim_{x\to 0}\frac{f^{\prime}(x)+xf^{\prime\prime}(x)-f^{\prime}(x)}{2x}\\
                    &=\lim_{x\to 0}\frac{f^{\prime\prime}}(x){2}=\frac{f^{\prime\prime}(x)}{2}=\frac{f^{\prime\prime}(0)}{2}=g^{\prime}(0)\\
                    &\therefore g^{\prime}(x)\mbox{在}x=0\mbox{处连续}
            \end{align*}
        %例题5
    \item 设$f(x)$在$(-\infty,+\infty)$上连续,$\lim\limits_{x \to 0}\frac{f(x)}{x}=A$,$A$为常数,$\varphi(x)=\int_0^1 f(xt)dt$,试讨论$\varphi^{\prime}(x)$的连续性。
        \begin{align*}
            sol:&\because\lim_{x\to 0}\frac{f(x)}{x}=A\quad\therefore f(0)=0\\
                &\therefore\varphi(0)=\int_0^1f(0)dt=0\\
                &x=0\mbox{时,}\varphi(x)\xrightarrow{u=xt}\varphi(x)=\int_0^xf(u)\frac{du}{x}=\frac{1}{x}\int_0^xf(u)du\\
                &\therefore\varphi^{\prime}(0)=\lim_{x\to 0}\frac{\frac{1}{x}\int_0^xf(u)du-0}{x}=\lim_{x\to 0}\frac{\int_0^xf(u)du}{x^2}=\lim_{x\to 0}\frac{f(x)}{2x}=\frac{A}{2}\\
                &x\neq 0\mbox{时,}\varphi^{\prime}(x)=\frac{f(x)x-\int_0^xf(u)du}{x^2}\\
                &\therefore\lim_{x\to 0}\varphi^{\prime}(x)=\lim_{x\to 0}\frac{f(x)x-\int_0^xf(u)du}{x^2}=\lim_{x\to 0}\frac{f(x)}{x}-\lim_{x\to 0}\frac{\int_0^xf(udu)}{x^2}\\
                &=A-\lim_{x\to 0}\frac{f(x)}{2x}=A-\frac{A}{2}=\frac{A}{2}\\
                &\therefore\varphi(x)\mbox{连续}
        \end{align*}
        %例题6
    \item 设$f(x)$在$(-\infty,+\infty)$上有定义,对任意$x$、$y$有$f(x+y)=e^xf(y)+e^yf(x)$,且$f(x)$在$x=0$处可导,$f^{\prime}(0)=2$,证明$f(x)$在任一点处可导,并求$f(x)$。
        \begin{align*}
            sol:&\mbox{令}x=y=0\Rightarrow f(0)=2f(0)\Rightarrow f(0)=0\\
                &\therefore f^{\prime}(x)=\lim_{\Delta x\to 0}\frac{f(x+\Delta x)-f(x)}{\Delta x}=\lim_{\Delta x\to 0}\frac{e^xf(\Delta x)+e^{\Delta x}f(x)-f(x)}{\Delta x}
        \end{align*}
        %例题7
    \item 设$f(x)$在$(-\infty,+\infty)$上连续,$f(1)=3$,对任意$x>0$,$y>0$有$\int_1^{xy}f(x)dt=x\int_1^y f(t)dt+y\int_1^x f(t)dt$,求$f(x)$。
        %例题8
    \item 设$f(x)=\begin{cases}x^3 \sin \frac{1}{x} & x\neq 0\\
                                0 & x=0
                \end{cases}$,求使得$f^{(n)}(0)$存在的最大$n$。
        %例题9
    \item 设$f(x)=3x^3 + x^2 \left|x\right|$,求使得$f^{(n)}(0)$存在的最大$n$。
        %例题10
    \item 设$y=(2x+3)^3(3x+2)^2$,求$y^{(5)}$,$y^{(6)}$。
        %例题11
    \item 设$y=\frac{x^4-x^3+2x^2-3x}{x-1}$,求$y^{(5)}$。
        %例题12
    \item 设$y=y(x)$由方程$y=\cos x +xe^y$所确定,求$y^{\prime}(0)$,$y^{\prime\prime}(0)$。
        %例题13
    \item 设$y=y(x)$由方程$2y^3-2y^2+2xy-x^2 =1$所确定,求$f(x)$的驻点,并判别该驻点是否为极值点。
        %例题14
    \item 通过变换$x=\sin t$化简方程$(1-x^2)\frac{d^2y}{dx^2}-x\frac{dy}{dx}+a^2 y=0$,并求原方程的通解。
        %例题15
    \item 通过变换$x=\frac{u}{\cos x}$化简方程$\cos x \frac{d^2 y}{d x^2}-2\sin x \frac{dy}{dx}+3y \cos x=e^x$,并求原方程的通解。
        %例题16
    \item 设$f(x)=y(x)$满足方程$y^{\prime\prime}+(x+e^{2y})y^{\prime 3}=0$,$y^{\prime}\neq 0$,试将该方程化为$y=y(x)$的反函数$x=x(y)$满足的微分方程,并求原方程的通解。
        %例题17
\end{enumerate}
% section 经典例题 (end)
